\documentclass{article}
\usepackage{enumitem}
\usepackage[utf8]{inputenc}
\usepackage[absolute]{textpos}
\TPGrid[1in, 0.25in]{23}{24}
\usepackage{palatino}
\parindent=0pt
\parskip=12pt
\usepackage{nopageno}

\begin{document}

\scriptsize

\begin{textblock}{23}(0,1)
\center\huge PREFACE
\end{textblock}

\begin{textblock}{11}(0, 3)

\section{}

    From \textbf{Volodimir Pavliuchuk}'s \emph{Cordial Waters}:

    \textbf{No.1 Plague Water} \emph{(1671, England)}

    \vspace{-\topsep}
    \begin{itemize}
        \renewcommand{\labelitemi}{$\circ$}
        \item 150 gm scabious \emph{(Scabiosa sp.)}
        \item 150 gm pimpernel \emph{(Anagallis arvensis)}
        \item 150 gm tormentil root \emph{(Potentilla erecta)}
        \item 4 litres 5\% malt extract wash\\
            [0.4\baselineskip]
            \emph{(strong beer as in the original)}
    \end{itemize}
    \vspace{-\topsep}

    Macerate for 12 hours and then distil.

    The recommended does is a spoonful every 4 hours.

    \textbf{No.2 Plague Water} \emph{(1677, England)}

    \vspace{-\topsep}
    \begin{itemize}
        \renewcommand{\labelitemi}{$\circ$}
        \item 100 gm rue
        \item 100 gm rosemary
        \item 100 gm sage
        \item 100 gm sorrel
        \item 100 gm celandine \emph{(Chelidonium majus)\\
            [0.4\baselineskip]
            (The leaves contain small amounts of toxic alkaloids\\
            which can be reduced greatly by drying the plant)}
        \item 100 gm mugwort \emph{(Artemisia vulgaris)}
        \item 100 gm bramble (blackberry) tops
        \item 100 gm pimpernel \emph{(Anagallis arvensis)}
        \item 100 gm dragons \emph{(Dracunculus vulgaris.)}
        \item 100 gm agrimony \emph{(Agrimonia eupatoria)}
        \item 100 gm lemonbalm
        \item 100 gm angelica leaves
        \item 4 litres white wine\\
            [0.4\baselineskip]
            \emph{(substitute a 15\% ABV sugar wash)}
    \end{itemize}
    \vspace{-\topsep}

    Macerate for 5 days and then distil.

\end{textblock}

\begin{textblock}{11}(12, 3)

\section{}

    \textbf{Baritone Saxophone}

    Bartok-pizzicato indications above noteheads indicate slap tongues.

    \textbf{Electric Guitar}

    The electric guitar should be treated with 6 different colors, via effects
    pedal(s). The color to be used is indicated at the beginning of each
    section of the score.  Pedal colors may include any combination of
    distortion, reverb or short delay (less than a quarter second).  A volume
    pedal should be placed last in the effect chain, to control overall
    dynamic.

    \textbf{Piano}

    Cross-shaped noteheads indicate glissandi on the tops of the keys, without
    depressing the keys, played with the flesh of the fingers, or fingernails.
    A flat or natural sign above the glissandi determines whether to play on
    the black or white keys.

    \textbf{Percussion}

    Instrumentation is somewhat up to the discretion of the performer, but
    should obey the following guidelines:

    \vspace{-\topsep}
    \begin{itemize}
        \item \emph{4 wooden shakers}, bamboo wind-chimes, maracas, rainsticks,
        cabasa, caxixi etc. These could include metal timbres, but should be
        primarily wood. The order of the shakers is not important.  Instruments
        with a longer decay, and a more granular sound quality, such as
        rainsticks and bamboo windchimes are preferred.
        \item \emph{5 wood blocks}, arranged from lowest to highest.  The
        exact pitch is not important. These could also be temple blocks. The
        sound quality should be very dry.
        \item \emph{3 large drums}, including at least one proper bass drum,
        arranged from lowest to highest.
    \end{itemize}
    \vspace{-\topsep}

    Percussion should be performed with bare hands. Wooden rings may be worn to
    increase the overall dynamic, especially on the wood blocks. Styrofoam
    blocks should be placed on the bass drums, to be used during the rehearsal
    marks indicated in the score (4, 14, 17a, 17b). Grace notes should always
    be played with the hands.

\end{textblock}

\end{document}