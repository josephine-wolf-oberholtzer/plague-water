\documentclass[11pt]{report}

\usepackage[T1]{fontenc}
\usepackage[papersize={17in, 11in}]{geometry}
\usepackage{tikz}
\usetikzlibrary{calc}
\usepackage{multicol}

\usepackage{xltxtra,fontspec,xunicode}
\defaultfontfeatures{Scale=MatchLowercase}
\setromanfont[Numbers=Uppercase]{Didot}

\parindent=0pt

\begin{document}
    \begin{titlepage}
    

        \begin{multicols}{3}
            \raggedcolumns
            \interlinepenalty=10000

            {
                \emph{A water against the plague\ldots}
                \vfill
            }

            \columnbreak

            {
                From Volodimir Pavliuchuk's \emph{Cordial Waters}:\\
            }

            \vspace*{2\baselineskip}

            {
                No.1 Plague Water \emph{(1671, England)}
                \vspace*{\baselineskip}
                \begin{itemize}
                    \renewcommand{\labelitemi}{$\circ$}
                    \item 150 gm scabious \emph{(Scabiosa sp.)}
                    \item 150 gm pimpernel \emph{(Anagallis arvensis)}
                    \item 150 gm tormentil root \emph{(Potentilla erecta)}
                    \item 4 litres 5\% malt extract wash\\
                        [0.4\baselineskip]
                        \emph{(strong beer as in the original)}
                \end{itemize}
                \vspace*{\baselineskip}
                Macerate for 12 hours and then distil.\\
                The recommended does is a spoonful every 4 hours.
            }

            \columnbreak
            \vspace*{3.2\baselineskip}

            {
                No.2 Plague Water \emph{(1677, England)}
                \vspace*{\baselineskip}
                \begin{itemize}
                    \renewcommand{\labelitemi}{$\circ$}
                    \item 100 gm rue
                    \item 100 gm rosemary
                    \item 100 gm sage
                    \item 100 gm sorrel
                    \item 100 gm celandine \emph{(Chelidonium majus)\\
                        [0.4\baselineskip]
                        (The leaves contain small amounts of toxic alkaloids\\
                        which can be reduced greatly by drying the plant)}
                    \item 100 gm mugwort \emph{(Artemisia vulgaris)}
                    \item 100 gm bramble (blackberry) tops
                    \item 100 gm pimpernel \emph{(Anagallis arvensis)}
                    \item 100 gm dragons \emph{(Dracunculus vulgaris.)}
                    \item 100 gm agrimony \emph{(Agrimonia eupatoria)}
                    \item 100 gm lemonbalm
                    \item 100 gm angelica leaves
                    \item 4 litres white wine\\
                        [0.4\baselineskip]
                        \emph{(substitute a 15\% ABV sugar wash)}
                \end{itemize}
                \vspace*{\baselineskip}
                Macerate for 5 days and then distil.
            }



        \end{multicols}
    \end{titlepage}
\end{document}
